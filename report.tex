    %\documentclass[bigbox]{jarticle}
    \documentclass{jarticle}[11pt]
    %\documentstyle[bigbox,fancybox]{jarticle}
     
    % コマンドの定義
    %
    % コメントアウト用のコマンド
    %   複数行にまたがる記述をまとめてコメントアウトする際に利用できる
    %   \COMMENT{ .... } で .... の部分をコメントアウト
    \newcommand{\COMMENT}[1]{}
     
    % 以下は,表(srmmary.tex)で使用しているコマンド
    \newcommand{\lw}[1]{\smash{\lower2.ex\hbox{#1}}}
     
    % 図を参照するためのマクロ
    \newcommand{\figref}[1]{\makebox{図~\ref{#1}}}
     
    % 表を参照するためのマクロ
    \newcommand{\tabref}[1]{\makebox{表~\ref{#1}}}

    %% 使用しているパッケージ等があれば,宣言しておく
    \usepackage{ascmac}
    \usepackage{graphicx} 
    \usepackage{afterpage}
    % 以下のパラメータは,見易いように適宜調整する.
    \topmargin=-1cm
    \textheight=24cm
    \textwidth=15.5cm
    \oddsidemargin=-.2cm
    \evensidemargin=-.2cm
     
    \title{{\normalsize 情報工学実験C(ソフトウェア)報告書}\\
    コンパイラ)\\
    } 
    \author{ 
      学生番号: 09429533 \\
      提出者: 高島和嗣
    }
    
    \date{
      提出日: 2020年 月日() \\ %% <-- 提出日を記載のこと
      締切日: 2020年 月日()
    }
    
    \begin{document}
    \maketitle
 %%%%%%%%%%%%%%%%%%%%%%%%%%%
\section{実験の目的}
\begin{itemize}
\item \verb|yacc|,\verb|lex|というプログラムジェネレータを使ってプログラムを作成する.
\item コンパイラを作成することでプログラム言語で書かれたプログラムとアセンブリ言語との対応について深く理解する.
\item 木構造の取り扱いを理解し,木構造を用いてC言語をアセンブリ言語に変換するコード生成のプログラムを作成する.
\end{itemize}

 %%%%%%%%%%%%%%%%%%%%%%%%%%%
\section{作成した言語の定義}
\begin{verbatim}
<プログラム> ::= <変数宣言部> <文集合>
<変数宣言部> ::= <宣言文> <変数宣言部> | <宣言文>
<宣言文> ::= define <識別子>; | array <識別子> [ <数> ];
<文集合> ::=  <文> <文集合>| <文>
<文> ::= <代入文> | <ループ文> | <条件分岐文>
<代入文> ::= <識別子> = <算術式>; | <識別子> [ <数> ] = <算術式>;
<算術式> ::= <算術式> + <項> | <算術式> - <項> | <項>
<項> ::= <項> * <因子> | <項> / <因子> | <因子>
<因子> ::= <変数> | (<算術式>)
<変数> ::= <識別子> | <数> | <識別子> [ <数> ]
<ループ文> ::= while (<条件式>) { <文集合> }
<条件分岐文> ::= if (<条件式>) { <文集合> } 
| if (<条件式>) { <文集合> } {<else文>} { <文集合> } 
| if (<条件式>) { <文集合> } {<else if文>} { <文集合> } 
| if (<条件式>) { <文集合> } {<else if文>} {<else文>} { <文集合> } 
<else文> ::= else { <文集合> }
<else if文> ::= else if (<条件式>) { <文集合> } 
| else if (<条件式>) { <文集合> } {<else if文>}
<条件式> ::= <算術式> == <算術式> | <算術式> < <算術式> | <算術式> > <算術式> 
<識別子> ::= <英字> <英数字列> | <英字>
<英数字列> ::= <英数字> <英数字列>| <英数字>
<英数字> ::= <英字> | <数字>
<数> ::= <数字> <数> | <数字>
<英字> ::= a|b|c|d|e|f|g|h|i|j|k|l|m|n|o|p|q|r|s|t|u|v|w|x|y|z|A|B|C|D|E|F|G|H|I|J|K
|L|M|N|O|P|Q|R|S|T|U|V|W|X|Y|Z
<数字> ::= 0|1|2|3|4|5|6|7|8|9
\end{verbatim}

 %%%%%%%%%%%%%%%%%%%%%%%%%%%
\section{定義した言語で受理されるプログラムの例}



 %%%%%%%%%%%%%%%%%%%%%%%%%%%
\section{コード生成の概略}
\subsection{メモリの使い方}
\subsection{レジスタの使い方}
\subsection{算術式のコード生成の方法}
 %%%%%%%%%%%%%%%%%%%%%%%%%%%
\section{工夫した点}
 %%%%%%%%%%%%%%%%%%%%%%%%%%%
\section{ソースプログラムのある場所}
\begin{verbatim}
/home/users/ecs/09429533/term3-4/compiler/lesson4/ast
\end{verbatim}
 %%%%%%%%%%%%%%%%%%%%%%%%%%%
\section{最終課題を解くために書いたプログラムの概要}
 %%%%%%%%%%%%%%%%%%%%%%%%%%%
\section{最終課題の実行結果}

\end{document}
